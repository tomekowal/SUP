\section{Introduction}

Fast development of mobile devices creates new domain of applications for large systems composed of many devices communicating over the Internet.
Those devices can be far from each other, what makes direct access difficult. Our system was designed to simplify software updates on groups of such devices.

There are many areas of applications including:
\begin{itemize}
    \item Monitoring - with our system we can perform software updates on cameras in the building. Especially if they are equiped with software for face recognition.
    \item Management of geographically spread devices including configuration management software update, position tracing, error detection.
	It can be applied to mobile phones and industrial robotics.
    \item Security - updates on sensors used in cars or alarms in homes.
    \item Robotics - autonomic robots used in factories and in industry.
\end{itemize}

General usecase of our system is to perform software updates on multiple devices over unreliable network. There is a need of such updates because of growing amount of remote devices such as sensors, mobile phones or network infrastructure devices. All of these devices need to run without breaks. That is where Erlang is useful with its updating mechanisms. We can perform software updates without interrupting running system and without the need to restart it.

 Existing solutions include Mobile Devices Management systems created by Nokia \cite{nokia}, Motorola \cite{motorola}, Oracle \cite{oracle} and by Open Mobile Alliance \cite{oma}.
