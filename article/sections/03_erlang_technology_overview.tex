\section{Erlang Technology Overview}

\subsection{Usability of Erlang Technology in distributed systems}

Erlang is a technology and a programming language that mixes functional programming with
an approach to easily build heavily parallel and distributed, highly available systems. It achieves
these goals using a set of unique features, including:

\begin{itemize}

\item Virtual machine implementing message-passing concurrency model with
lightweight Erlang processes
\item Built-in, language-integrated engine for communication in a distributed environment
\item Hot code swapping with a fine control over the software upgrade process. The aim of these
is to allow an upgrade to be performed automatically without stopping any services.
\item Fault-tolerance features like supervisors.
\item Takeover and failover mechanisms for cluster systems.

\end{itemize}

\subsection{General description of Erlang OTP}

Erlang OTP (Open Telecom Platform) is an Erlang distribution released by Ericsson in 1998
when the language became open source. OTP is a set of standard erlang libraries and
corresponding, well-defined design principles for Erlang developers. OTP defines patterns
for basic elements that make up the software as well as the general layout of completed,
deployed environment.

OTP principles include:
\begin{itemize}
\item Supervisors and supervision trees

Erlang software can be thought of as a set of lightweight erlang processes communicating with
each other. In supervision tree principle, these processes form a tree where the leaves are
called workers and are doing the actual job, while other nodes are called supervisors.
Each supervisor is responsible for monitoring its children and reacting accordingly when
any of them crashes. Supervisors allow to design well-structured and fault-tolerant
software.

\item Behaviours

Behaviours are a set of basic design patterns used to build common types of software
pieces. Fundamental behaviours are:
\begin{itemize}
\item {\tt gen\_server} for implementing simple servers and client-server relation between erlang
processes
\item {\tt gen\_fsm} for implementing generic finite state machines
\item {\tt gen\_event} for implementing event handling subsystems
\end{itemize}
\item Applications and releases

These patterns define general layout of a self-contained, deployable piece of Erlang
software. Applications and releases will be described in the next section as the
Software Update Platform deals heavily with them.

\end{itemize}

\subsection{Applications and releases in Erlang}

An example of deployable package of software written in Erlang is so-called embedded
node. An embedded node is a self-contained, configured Erlang environment along with actual software
written in Erlang that can be deployed and run using a simple command. An embedded node contains:
\begin{itemize}
\item Erlang Runtime System
\item A set of Erlang applications, the actual code
\item Configuration for ERTS and applications
\item An Erlang release
\end{itemize}

\subsubsection{Erlang applications}

An erlang application is an independent piece of software that serves some particular
functionality. An application is defined by its name, version, code (set of modules),
dependencies (other applications) and other more finegrained settings and attributes. These are all
configured in an {\tt .app} file. Every application defines a way of starting it, stopping it and possibly
upgrading or downgrading it to another version (optional {\tt appup} file). It also has its own piece of configuration.
A running application is often made up of a single supervision tree.

\subsubsection{Erlang release}

An erlang release is a configuration of what an embedded node contains and how it is
started, stopped and upgraded. Thus, an erlang release, defined by an {\tt .rel} file,
states which version of ERTS should be used in the node, lists a set of applications in 
particular versions that should be part of the release, and defines one or more way the
Erlang node is started and stopped. Separate, optional {\tt relup} file defines how the
release is upgraded or downgraded (without stopping the node).
Erlang release has its own version number.

\subsection{Upgrading Erlang software}

As a part of focus for high available systems, Erlang supports hot code swapping and
very finegrained control over the upgrade process without stopping running node.
Every application may define how its version should be changed to higher or lower in the
{\tt appup} file. Based on a set of {\tt appup} files, a {\tt relup} file may be generated
which merges all operations listed in {\tt appup} files into one big script that
upgrades or downgrades the whole release. This script may be executed using standard erlang
API for release handling (the {\tt release\_handler} module).

Because each application is responsible
for defining how its version should be changed, the upgrade process is very straightforward
and requires only a few calls to the release handling functions. Thus, it can be easily
performed without human interaction, by automatic tools.

Still, there are some problems with that method. Mainly, it is low-level as it requires
calling the erlang API on the target node. What is needed and what our platform aims to provide is
a way of easy installation and deinstallation of the erlang node on the target system using
simple tools like package managers and also a way of easy management of a large number of devices.

