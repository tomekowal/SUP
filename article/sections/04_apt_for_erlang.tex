\section{Packaging of erlang software}

\subsection{What it is about and motivation for it}

By `packaging of erlang software`, we mean a way of putting the contents of erlang node
into packages like {\tt .deb}. These would be easy to install, upgrade and remove with standard
package manager commands. We also want to maintain ability to perform upgrades without
stopping the node (hot code swapping).

There are a few reasons why we decided to implement this:

\begin{itemize}
\item Easy installation and deinstallation of erlang node on the target system using a single command.
\item Reduction of amount of data downloaded during the upgrade (only the actually changed packages
are fetched by the package manager).
\item Overall better integration with the target system.
\end{itemize}

\subsection{How it was implemented}

We decided to create tools to build debian packages {\tt .deb} that would span the contents
of the erlang node and create a set of packages ready to be pushed to a repository and easily installed
on the target device. This also required general implementation of debian maintainer scripts
included into these packages. These scripts are responsible for management of the node
during installation. Most importantly, they contain the code that performs the upgrade process.

\subsubsection{Decomposition of the erlang node into debian package set}

The erlang node is decomposed into a set of debian packages in the following way:
\begin{itemize}
\item Base package contains ERTS and its version is equal to the ERTS version. This package is
architecture-dependent. It has no dependencies.
\item Each erlang application gets its own package. Its version is equal to the application version.
Package dependencies reflect the list of dependent applications in the {\tt .app} file.
\item Erlang release files are packaged into the final, main package. This package is dependent on the base
package and packages for all applications contained in the release. These dependencies are strict in terms
of version of dependent packages - the release requires a concrete version of every application as well as the ERTS.
\end{itemize}

The only package that should be explicitly maintained by system administrator is the main package containing the release files.
This package, thanks to its well-defined dependencies, represents the whole erlang node and its installation will cause
the whole node to be deployed on the device.

The point of splitting up the node into several packages related through dependencies was to reduce the amount
of data downloaded during the upgrade process. It is a very common situation when we want to upgrade e.g. only
one erlang application. Standard erlang features do not allow us to independently upgrade each application - we must
always create a new version of the whole release and upgrade the release itself. Without proper decomposition of the contents
of the release, this would force us to download a big package containing every application, even though most of them did not change.

Usage of package manager solves this problem. When the node is decomposed into several packages, upgrade of the main package
forces the package manager to download only these application packages whose version changed. This is all thanks to automatic
dependency resolution provided by the package manager.

\subsubsection{Upgrade of the release - maintainer scripts' job}

There are a few maintainer scripts in all the packages spanning the erlang node.
They are responsible for starting the node when it gets installed, stopping it when it gets
removed and, most importantly, performing the release upgrade process when the main package is being
upgraded by the package manager.

The script performing the release upgrade checks whether the node is running, using tools available in the node {\tt bin} directory.
When the node is running, the script remotely calls appropriate erlang code on the running node, causing it to switch to the higher
version of the release (standard erlang release upgrade using the {\tt release\_handler}, without stopping the node).
Hot upgrade may fail or the node may have been down from the beginning. If so, a manual replacement of the old version of the release with the new version is performed (manual replacement of some files) and the node is started.

\subsection{Results and effects} 
\subsection{Problems and limitations}
