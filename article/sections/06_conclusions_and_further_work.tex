\section{Conclusions and Further Work}

We managed to solve most of occuring problems. Our system is fully functional. Even though \emph{apt} does not provide simple solution to hot upgrades, we implemented it. This gave us some advantages over standard tools like \emph{rebar}. We optimized amount of data to download. Using package manager to install and upgrade Erlang applications is easier for people who had never used Erlang before.

\begin{itemize}
\item Support for other package managers

One obvious improvement for our platform would be support for more package managers, as
debian package format {\tt .deb} was chosen arbitrarily for experimental development.
Since its integration was successful, support for {\tt rpm}-based package managers (like {\tt yum}),
{\tt pacman} or {\tt port} is planned in the future.

\item Communication security

Right now, communication between devices and the server is not encrypted. Wrapping device-server
sessions in TLS and adding some authentication is another important potential feature for our
platform.

\item Development towards more general management platform

The protocol used in communication between devices and the server is very general
and extensible. New types of jobs can be easily added. Thus, the platform can be turned into
more general management platform, providing a lot more of different functionality:

\begin{itemize}
\item monitoring of the devices, their status, statistics, etc.
\item configuration of software installed on the device
\item maintenance features (viewing logs, etc.)
\item peforming diagnostics, troubleshooting, tests etc.
\end{itemize}

\end{itemize}

