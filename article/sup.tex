\documentclass[10pt]{article}

\usepackage{amsmath} \allowdisplaybreaks
\usepackage{amssymb}

\usepackage[T1]{fontenc}
\usepackage[polish,english]{babel}
\usepackage[utf8]{inputenc}
\usepackage{lmodern}

\usepackage[en]{csagh}
\usepackage{graphics}
\usepackage{graphicx}

\makeatletter

\begin{document}

\begin{opening}

\tytulPL{PLATFORMA DO AKTUALIZACJI OPROGRAMOWANIA NA URZĄDZENIACH MOBILNYCH OPARTA NA TECHNOLOGII ERLANG}

\titleEN{ERLANG-BASED SOFTWARE UPDATE PLATFORM FOR MOBILE DEVICES}

\author{Małgorzata Wielgus\affiliation
            {AGH University of Science and Technology,
             Kraków, Poland,
             \texttt{malgorza@student.agh.edu.pl}},
        Przemysław Dąbek\affiliation
           {AGH University of Science and Technology,
             Kraków, Poland,
             \texttt{przemyslaw.dabek@gmail.com}},
        Roman Janusz\affiliation
           {AGH University of Science and Technology,
             Kraków, Poland,
             \texttt{roman@student.agh.edu.pl}},
        \\Tomasz Kowal\affiliation
           {AGH University of Science and Technology,
             Kraków, Poland,
             \texttt{tomekowal@gmail.com}},
        Wojciech Turek\affiliation
            {AGH University of Science and Technology,
             Kraków, Poland,
             \texttt{wojciech.turek@agh.edu.pl}}}

%\date{Kraków, 15 wrzesśnia 2004}

\begin{streszczenie}
To pisze się na samym końcu

\slowakluczowe {Erlang, aktualizacja oprogramowania, urządzenia przenośne}
\end{streszczenie}

\begin{abstract}
To be written at the very end

\keywords{Erlang, software updates, mobile devices}
\end{abstract}

\end{opening}


% SECTIONS

\section{Introduction}

\begin{itemize}
	\item intro: Fast development of mobile devices creates new domain of applications for large systems composed of loosely ...
	\item Applications: monitoring, management, security, robotics (4 paragraphs)
	\item Usability of Erlang Technology in distributed systems -- general features, advantages
	\item The need: management and software updates, use cases 
	\item Existing solutions (several paragraphs, citations \cite{asoc}) 
	\item ``The tool described in this paper ....... ``
\end{itemize}

\section{Requirements and assumptions}

During system implementation we made following assumptions:
\begin{itemize}
  \item Possibility of NAT -- usually mobile devices we want to update are not in the local network. This implies that those devices are behind a NAT. It makes ssh connection difficult. Devices should connect to central server with public IP.
  \item Slow and unreliable connections -- while developing we thought mainly about devices that are connected via GSM. This connection can be slow and may be interrupted easily.
  \item Need for scalability -- distributed systems has to be prepared for heavy load because they consist of large amount of devices. Our system should be able to manage up to about 10000 devices.
  \item Simple, intuitive interface -- we wanted our system to be user-friendly.
  \item Managing groups of devices, batch updates -- user can create groups of devices so that our system can be used to manage more than one distributed system at once. User can send requests to those groups. He alse has fine-grained control over particular device.
\end{itemize}

\section{Erlang Technology Overview}

\subsection{Usability of Erlang Technology in distributed systems}

Erlang is a technology and a programming language that mixes functional programming with
an approach to easily build heavily parallel and distributed, highly available systems. It achieves
these goals using a set of unique features, including:

\begin{itemize}

\item Virtual machine implementing message-passing concurrency model with
lightweight Erlang processes
\item Built-in, language-integrated engine for communication in a distributed environment
\item Hot code swapping with a fine control over the software upgrade process. The aim of these
is to allow an upgrade to be performed automatically without stopping any services.
\item Fault-tolerance features like supervisors.
\item Takeover and failover mechanisms for cluster systems.

\end{itemize}

\subsection{General description of Erlang OTP}

Erlang OTP (Open Telecom Platform) is an Erlang distribution released by Ericsson in 1998
when the language became open source. OTP is a set of standard erlang libraries and
corresponding, well-defined design principles for Erlang developers. OTP defines patterns
for basic elements that make up the software as well as the general layout of completed,
deployed environment.

OTP principles include:
\begin{itemize}
\item Supervisors and supervision trees

Erlang software can be thought of as a set of lightweight erlang processes communicating with
each other. In supervision tree principle, these processes form a tree where the leaves are
called workers and are doing the actual job, while other nodes are called supervisors.
Each supervisor is responsible for monitoring its children and reacting accordingly when
any of them crashes. Supervisors allow to design well-structured and fault-tolerant
software.

\item Behaviours

Behaviours are a set of basic design patterns used to build common types of software
pieces. Fundamental behaviours are:
\begin{itemize}
\item {\tt gen\_server} for implementing simple servers and client-server relation between erlang
processes
\item {\tt gen\_fsm} for implementing generic finite state machines
\item {\tt gen\_event} for implementing event handling subsystems
\end{itemize}
\item Applications and releases

These patterns define general layout of a self-contained, deployable piece of Erlang
software. Applications and releases will be described in the next section as the
Software Update Platform deals heavily with them.

\end{itemize}

\subsection{Applications and releases in Erlang}

An example of deployable package of software written in Erlang is so-called embedded
node. An embedded node is a self-contained, configured Erlang environment along with actual software
written in Erlang that can be deployed and run using a simple command. An embedded node contains:
\begin{itemize}
\item Erlang Runtime System
\item A set of Erlang applications, the actual code
\item Configuration for ERTS and applications
\item An Erlang release
\end{itemize}

\subsubsection{Erlang applications}

An erlang application is an independent piece of software that serves some particular
functionality. An application is defined by its name, version, code (set of modules),
dependencies (other applications) and other more finegrained settings and attributes. These are all
configured in an {\tt .app} file. Every application defines a way of starting it, stopping it and possibly
upgrading or downgrading it to another version (optional {\tt appup} file). It also has its own piece of configuration.
A running application is often made up of a single supervision tree.

\subsubsection{Erlang release}

An erlang release is a configuration of what an embedded node contains and how it is
started, stopped and upgraded. Thus, an erlang release, defined by an {\tt .rel} file,
states which version of ERTS should be used in the node, lists a set of applications in 
particular versions that should be part of the release, and defines one or more way the
Erlang node is started and stopped. Separate, optional {\tt relup} file defines how the
release is upgraded or downgraded (without stopping the node).
Erlang release has its own version number.

\subsection{Upgrading Erlang software}

As a part of focus for high available systems, Erlang supports hot code swapping and
very finegrained control over the upgrade process without stopping running node.
Every application may define how its version should be changed to higher or lower in the
{\tt appup} file. Based on a set of {\tt appup} files, a {\tt relup} file may be generated
which merges all operations listed in {\tt appup} files into one big script that
upgrades or downgrades the whole release. This script may be executed using standard erlang
API for release handling (the {\tt release\_handler} module).

Because each application is responsible
for defining how its version should be changed, the upgrade process is very straightforward
and requires only a few calls to the release handling functions. Thus, it can be easily
performed without human interaction, by automatic tools.

Still, there are some problems with that method. Mainly, it is low-level as it requires
calling the erlang API on the target node. What is needed and what our platform aims to provide is
a way of easy installation and deinstallation of the erlang node on the target system using
simple tools like package managers and also a way of easy management of a large number of devices.


\section{Packaging of erlang software}

\subsection{What it is about and motivation for it}

By `packaging of erlang software`, we mean a way of putting the contents of erlang node
into packages like {\tt .deb}. These would be easy to install, upgrade and remove with standard
package manager commands. We also want to maintain ability to perform upgrades without
stopping the node (hot code swapping).

There are a few reasons why we decided to implement this:

\begin{itemize}
\item Easy installation and deinstallation of erlang node on the target system using a single command.
\item Reduction of amount of data downloaded during the upgrade (only the actually changed packages
are fetched by the package manager).
\item Overall better integration with the target system.
\end{itemize}

\subsection{How it was implemented}

We decided to create tools to build debian packages {\tt .deb} that would span the contents
of the erlang node and create a set of packages ready to be pushed to a repository and easily installed
on the target device. This also required general implementation of debian maintainer scripts
included into these packages. These scripts are responsible for management of the node
during installation. Most importantly, they contain the code that performs the upgrade process.

\subsubsection{Decomposition of the erlang node into debian package set}

The erlang node is decomposed into a set of debian packages in the following way:
\begin{itemize}
\item Base package contains ERTS and its version is equal to the ERTS version. This package is
architecture-dependent. It has no dependencies.
\item Each erlang application gets its own package. Its version is equal to the application version.
Package dependencies reflect the list of dependent applications in the {\tt .app} file.
\item Erlang release files are packaged into the final, main package. This package is dependent on the base
package and packages for all applications contained in the release. These dependencies are strict in terms
of version of dependent packages - the release requires a concrete version of every application as well as the ERTS.
\end{itemize}

The only package that should be explicitly maintained by system administrator is the main package containing the release files.
This package, thanks to its well-defined dependencies, represents the whole erlang node and its installation will cause
the whole node to be deployed on the device.

The point of splitting up the node into several packages related through dependencies was to reduce the amount
of data downloaded during the upgrade process. It is a very common situation when we want to upgrade e.g. only
one erlang application. Standard erlang features do not allow us to independently upgrade each application - we must
always create a new version of the whole release and upgrade the release itself. Without proper decomposition of the contents
of the release, this would force us to download a big package containing every application, even though most of them did not change.

Usage of package manager solves this problem. When the node is decomposed into several packages, upgrade of the main package
forces the package manager to download only these application packages whose version changed. This is all thanks to automatic
dependency resolution provided by the package manager.

\subsubsection{Upgrade of the release - maintainer scripts' job}

There are a few maintainer scripts in all the packages spanning the erlang node.
They are responsible for starting the node when it gets installed, stopping it when it gets
removed and, most importantly, performing the release upgrade process when the main package is being
upgraded by the package manager.

The script performing the release upgrade checks whether the node is running, using tools available in the node {\tt bin} directory.
When the node is running, the script remotely calls appropriate erlang code on the running node, causing it to switch to the higher
version of the release (standard erlang release upgrade using the {\tt release\_handler}, without stopping the node).
Hot upgrade may fail or the node may have been down from the beginning. If so, a manual replacement of the old version of the release with the new version is performed (manual replacement of some files) and the node is started.

\subsection{Results and effects} 
\subsection{Problems and limitations}

\section{Software Update Platform}

General notes: aim of the system... 


\subsection{Architecture}

\begin{itemize}
	\item Description of general architecture 
	\item figure -- general architecture 
	\item Description of particular components 
	\item Cooperation of components, 
	\item figures -- most interesting sequences  
	\item Explained... 
	\item Technology, 
\end{itemize}


\subsection{Functionality}

\begin{itemize}
	\item General description of the application: Web interface, technology 
\end{itemize}

gui: 1 or 2 images 


\section{Conclusions and Further Work}

\begin{itemize}
\item Support for other package managers
\item Communication security
\item Development towards more general management platform
\begin{itemize}
\item monitoring
\item configuration
\item maintenance
\item diagnostics
\end{itemize}
\end{itemize}



% BIBLIOGRAPHY
\begin{thebibliography}{99}

\bibitem{nokia}
\emph{Nokia Device management}
http://europe.nokia.com/find-products/nokia-for-business/device-management

\bibitem{motorola}
\emph{Motorola Mobility Services Platform (MSP)}
http://www.motorola.com/Business/US-EN/Business+Product+and+Services/Software+and+Applications/Mobility+Software\\
/Mobile+Device+Management+Software/Mobility+Services+Platform\_US-EN

\bibitem{oracle}
\emph{Oracle Database Mobile Server 11g}
http://www.oracle.com/technetwork/database/database-mobile-server/overview/index.html

\bibitem{oma}
\emph{Open Mobile Alliance}
http://www.openmobilealliance.org/

\bibitem{otp}
\emph{OTP Design Principles}
http://www.erlang.org/doc/design\_principles/users\_guide.html

\bibitem{maintainerscripts}
\emph{Debian Policy Manual - Package maintainer scripts and installation procedure}
http://www.debian.org/doc/debian-policy/ch-maintainerscripts.html

\bibitem{rebar}
\emph{Rebar: Erlang Build Tool}
https://github.com/basho/rebar/wiki

\end{thebibliography}

\end{document}
