% By zmienic jezyk na angielski/polski, dodaj opcje do klasy english lub polish
\documentclass[polish,12pt]{aghthesis} \usepackage[utf8]{inputenc}
\usepackage{url} \usepackage{listings}

\author{Przemysław Dąbek, Roman Janusz\\ Tomasz Kowal, Małgorzata Wielgus}

\title{Platforma do automatycznych aktualizacji oprogramowania na urządzeniach
  zdalnych. - dokumentacja procesu}

\supervisor{dr inż. Wojciech Turek}

\date{2012}

% Szablon przystosowany jest do druku dwustronnego,
\begin{document}

\maketitle

\section{Podział na milestony}
Prace zostały podzielone na dwa milestony. Zakończenie pierwszego miało na celu zapoznanie z technologiami, które można wykorzystać do rozwiązania problemu automatycznej aktualizacji oprogramowania. Celem Milestone 1 jest stworzenie wizji oraz koncepcji architektury systemu, a także przetestowanie branych pod uwagę technologii. Produktem końcowym tego etapu były:
\begin{itemize}
\item Prototyp komunikacji serwera z klientem
\item Zapoznanie z działaniem menedżerów pakietów \emph{apt} oraz \emph{opkg}
\item Wybór technologii do stworzenia graficznego interfejsu użytkownika
\item Wybór bazy danych
\end{itemize}

Drugi milestone zakończył się dostarczeniem działającego produktu.
\clearpage

\subsection{Milestone 1}
\begin{tabular}{| p{1.3cm} | p{5cm} | p{2.5cm} | p{1.5cm} | p{1.5cm} | p{1.5cm} |}
\hline
Ticket & Summary & Type & Owner & Status & Created \\ \hline
\#273 & Wizja - określenie wymagań & task & tkowal & closed & 06/05/11 \\ \hline
\#274 & Wizja - wstępna analiza ryzyka & task & szemek & closed & 06/05/11 \\ \hline
\#275 & Wizja - ogólny opis produktu & task & tkowal & closed & 06/05/11 \\ \hline
\#276 & Przegląd technologii apt & task & szemek & closed & 06/05/11 \\ \hline
\#277 & Przegląd technologii bazodanowych. & task & szemek & closed & 06/05/11 \\ \hline
\#278 & Przegląd technologii do tworzenia GUI & task & szemek & closed & 06/05/11 \\ \hline
\#279 & Przegląd technologii do komunikacji ze zdalnymi urządzeniami & task & tkowal & closed & 06/05/11 \\ \hline
\#280 & Określenie architektury systemu. & task & malgorza & closed & 06/05/11 \\ \hline
\#281 & Przetestowanie komunikacji z użyciem modułu gen\_tcp. & task & tkowal & closed & 06/05/11 \\ \hline
\#282 & Przetestowanie możliwości stworzenia GUI przy pomocy eHTML. & task & szemek & closed & 06/05/11 \\ \hline
\#283 & Połączenie erlanga z bazą danych - drivery MySQL i PostgreSQL & task & szemek & closed & 06/05/11 \\ \hline
\#285 & Sprawdzenie możliwości wykorzystania menedżera pakietów OPKG & task & tkowal & closed & 06/20/11 \\ \hline
\#327 & Rozważenie możliwości użycia i przetestowanie mochiweb & task & tkowal & closed & 07/22/11 \\ \hline
\#284 & Instalacja Debiana na beagleboardzie & task & roman & closed & 06/20/11 \\ \hline
\end{tabular}
\clearpage

\subsection{Milestone 2}
\begin{tabular}{| p{1.3cm} | p{5cm} | p{2.5cm} | p{1.5cm} | p{1.5cm} | p{1.5cm} |}
\hline
Ticket & Summary & Type & Owner & Status & Created \\ \hline
\#334 & GUI do przeglądania urządzeń i dodawania prostych zadań & enhancement & malgorza & closed & 10/19/11 \\ \hline
\#333 & Integracja ERLRC & task & roman & closed & 10/19/11 \\ \hline
\#331 & Zmniejszenie rozmiaru paczek z releasami generowanych przez rebara & enhancement & roman & closed & 09/27/11 \\ \hline
\#328 & Problem z uruchomieniem releasu wygenerowanego rebarem & defect & tkowal & closed & 09/27/11 \\ \hline
\#329 & Implementacja infrastruktury sesji zarządania urządzeniem na kliencie i serwerze & enhancement & roman & closed & 09/27/11 \\ \hline
\#286 & Zapoznanie się z OTP Design Principles (application, release) oraz narzędziem rebar & task & roman & closed & 06/20/11 \\ \hline
\#330 & Instalacja i integracja z systemem serwera FTP, implentacja prostego klienta FTP dla urządzenia. & enhancement & roman & closed & 09/27/11 \\ \hline
\#332 & Mechanizm logowania & enhancement & szemek & closed & 10/18/11 \\ \hline
\end{tabular}

\section{Podział prac}
W trakcie prac podzieliliśmy się na 2 podzespoły: Przemysław Dąbek i Małgorzata Wielgus zajmowali się front-endem, natomiast Roman Janusz i Tomasz Kowal zajmowali się backendem. Podział nie był sztywny i często pracowaliśmy wspólnie w czwórkę.

\subsection{Prace wykonane przez poszczególne osoby}

\begin{itemize}
  \item Roman Janusz
  \begin{itemize}
    \item Dokładne zapoznanie się z wzorcami OTP tworzenia oprogramowania w Erlangu - w szczególności \emph{application} i \emph{release}
    \item Zaprojektowanie ogólnej struktury oprogramowania przeznaczonego na urządzenia zdalne. Opracowanie modelu rozwoju oprogramowania z użyciem narzęcia \emph{rebar} i dodatkowych skryptów.
    \item Dokładne zapoznanie się ze strukturą pakietów \emph{deb}, działaniem menedżera pakietów \emph{dpkg}/\emph{apt} oraz metodami tworzenia pakietów i instalacji repozytorium. Rozpoznanie możliwości paczkowania oprogramowania tworzonego w erlangu z zachowaniem możliwości wykonywania hot-upgrade podczas aktualizacji pakietu.
    \item Zaprojektowanie sposobu dekompozycji erlangowego release'u w zestaw pakietów \emph{deb}. Stworzenie ogólnych skryptów (debian maintainer scripts) używanych podczas instalacji, aktualizacji i usuwania pakietów. Obsługa mechanizmu hot-upgrade podczas aktualizacji.
    \item Implementacja skryptów użytkownika do tworzenia pakietów \emph{deb} na podstawie erlangowego release'u wraz z ich konfiguracją.
    \item Integracja mechanizmu menedżera pakietów z platformą do automatycznych aktualizacji.
    \item Dokumentacja użytkownika i techniczna dotycząca sposobu wykorzystania menedżera pakietów \emph{deb} w projekcie.
  \end{itemize}

  \item Tomasz Kowal
  \begin{itemize}
    \item Testowanie możliwości serwera Mochiweb (ostatecznie użytego jako serwer http).
    \item Zbadanie różnych możliwości komunikacji klienta i backendu (\emph{gen\_tcp}, \emph{gen\_rcp}, użycie \emph{JSON}).
    \item Implementacja serwera przyjmującego zgłaszające się urządzenia klienckie.
    \item Szkielet niskopoziomowej komunikacji.
  \end{itemize}
  \item Przemysław Dąbek
    \begin{itemize}
      \item Przygotowanie interfejsu do operacji na bazie danych (mnesia).
      \item Zaprojektowanie webowego interfejsu użytkownika oraz logo.
      \item Implementacja logiki odpowiedzialnej za przetwarzanie żądań HTTP.
      \item Dodanie mechanizmu logowania (lager).
    \end{itemize}
  \item Małgorzata Wielgus
    \begin{itemize}
      \item Stworzenie prototypu interfejsu graficznego przy użyciu serwera \emph{Yaws}.
      \item Integracja backendu i frontendu serwera zarządzającego.
      \item Rozszerzenie interfejsu graficznego o funkcjonalność zlecania zadań urządzeniom.
      \item Testowanie różnych scenariuszy zakończenia \emph{joba}.
    \end{itemize}
\end{itemize}

\section{Spotkania z klientem}
\subsection{Treść notatki z 2011-06-07}
Zapoznać się:
\begin{itemize}
\item RPC w Erlangu \url{http://erldocs.com/R14B02/kernel/rpc.html}
\item rebar \url{https://bitbucket.org/basho/rebar/wiki/Home}
\item release, update aplikacji erlangowych
\end{itemize}
Do zrobienia:
\begin{itemize}
\item prototyp technologiczny (repozytorium, urządzenie mobilne, serwer http, komunikacja pomiędzy składowymi systemu)
\item opracowanie harmonogramu
\end{itemize}
Pomysły rozbudowy funkcjonalności:
\begin{itemize}
\item GPS, Google Maps
\item VM i aplikacja uruchamiane przy starcie systemu na urządzeniu mobilnym
\end{itemize}
\subsection{Treść notatki z 2011-09-29}
W czasie spotkania klient ocenił nasze dotychczasowe postępy.

Do zrobienia:
\begin{itemize}
\item integracja poszczególnych części projektu
\item integracja systemu z menedżerami pakietów
\end{itemize}

\subsection{Treść notatki z 2011-10-28}
W czasie spotkania odbyła się prezentacja prototypu

Klient zgłosił następujące uwagi:
\begin{itemize}
  \item apt - dokończyć generator
  \item apt - integracja z systemem aktualizacji
  \item pobieranie pakietów z http - opcjonalnie
  \item interfejs --- dokończenie
    \begin{itemize}
    \item edycja grupowa
    \item zakończone zadania
    \item stan - uruchomione aplikacje, wersje
    \item MAC - identyfikacja
    \end{itemize}
\end{itemize}

\section{Przegląd technologii do zastosowania w platformie}
\subsection{Technologia do stworzenia graficznego interfejsu użytkownika}
Pod uwagę brano:
\begin{itemize}
\item Ruby on Rails z połączeniem Erlectricity
\item Yaws
\item Mochiweb
\item Webmachine
\item Nitrogen
\item Zotonic – CMS i framework
\end{itemize}

RoR: Według przykładów najpierw uruchamiany był proces erlanga, który dopiero wywoływał program w Rubym i dopiero wtedy zachodziła komunikacja. Nie można jednym procesem erlangowym dopiąć się do jednej klasy aplikacji w Ruby on Rails.

Yaws: Serwer napisany całkowicie w Erlangu. Tworzenie interfejsu odbywa się w specjalnym dialekcie eHTML. Wydaje się być najlepszym rozwiązaniem.

Mochiweb: jest narzędziem do budowania własnych lekkich serwerów http. Zbudowanie własnego serwera od zera dodałoby niepotrzebny dodatkowy stopień do złożoności problemu.

Webmachine: Zestaw narzędzi do tworzenia web serwisów opartych o technologię REST. Prawdopodobnie nie będzie nam potrzebny.

Nitrogen: Jest to framework do tworzenia aplikacji webowych w erlangu. Do tego potrzebny byłby jeszcze serwer.

Zotonic: Framework i CMS w erlangu. Aby dopisać obsługę serwera trzeba napisać dodatkowe moduły i poznać jego strukturę.

\subsection{Komunikacja między urządzeniem zdalnym, a głównym serwerem platformy:}
\begin{itemize}
\item wywoływanie poleceń przez SSH
\item komunikacja przez socket TCP
\item komunikacja między węzłami sieci erlanga
\item web service
\end{itemize}

SSH: Połączenie może być w każdej chwili zrywane, co może skutkować nieprzewidywalnym zachowaniem. Wymaga również zmian konfiguracji w systemie operacyjnym zdalnego urządzenia (użytkownik, klucz, plik sudoers).

TCP: Rozwiązanie o najmniejszym narzucie komunikacyjnym. Łatwe do obsłużenia w erlangu za pomocą gen\_tcp. Wymagana samodzielna obsługa zrywanych połączeń, zaprojektowanie własnego formatu wiadomości. Wydaje się być najlepszym rozwiązaniem.

Węzły sieci erlanga: Po połączeniu z danym urządzeniem połączenie jest stale utrzymywane, co nie jest pożądane (ani nawet możliwe) w naszym przypadku.

Web Serwisy: Powodują duży narzut komunikacyjny, a ponieważ zakładamy dużą liczbę urządzeń, chcieliśmy tego uniknąć.

\subsection{Język programowania do implementacji klienta oraz serwera}
W związku z tym, że wybraliśmy komponenty, które są napisane w erlangu, to aby uniknąć problemów na styku różnych technologii, postanowiliśmy napisać zarówno klienta, jak i serwer w erlangu.

\subsection {Wybór bazy danych}
\begin{itemize}
\item mnesia
\item MySQL
\item ewentualnie inne relacyjne bazy danych
\end{itemize}


\end{document}
