% By zmienic jezyk na angielski/polski, dodaj opcje do klasy english lub polish
\documentclass[polish,12pt]{aghthesis}
\usepackage[utf8]{inputenc}
\usepackage{url}
\usepackage{listings}

\author{Przemysław Dąbek, Roman Janusz\\ Tomasz Kowal, Małgorzata Wielgus}

\title{Platforma do automatycznych aktualizacji oprogramowania na urządzeniach zdalnych - Przewodnik}

\supervisor{dr inż. Wojciech Turek}

\date{2012}

% Szablon przystosowany jest do druku dwustronnego,
\begin{document}

\maketitle

\section{Cel prac i wizja produktu}
\subsection{Opis problemu}
Problem: Duża ilość zdalnych urządzeń rozmieszczonych w terenie. Urządzenia komunikują się przez zawodną sieć. Na różnych urządzeniach działają różne wersje aplikacji. Zachodzi potrzeba aktualizacji oprogramowania na grupie urządzeń. Rozwiązanie: Platforma umożliwi monitorowanie oraz aktualizację wersji aplikacji na grupach urządzeń.

\subsection{Opis użytkownika i zewnętrznych podsystemów}
Użytkownikiem systemu jest osoba odpowiedzialna za oprogramowanie w systemie składającym się z urządzeń zdalnych. System składa się z dużej liczby tych urządzeń, które mogą mieć mocno ograniczone zasoby sprzętowe (np. \emph{Beagleboard}). Urządzenia mogą mieć różne architektury. Urządzenia będą łączyć się z serwerem głównym, do którego przesyłają zebrane dane. Komunikacja odbywa się za pomocą zawodnego połączenia (np. \emph{GSM}).

\subsection{Opis produktu}
Produkt składa się z serwera zawierającego repozytorium oprogramowania. Serwer ma za zadanie monitorować jakie wersje aplikacji znajdują się na konkretnych urządzeniach i grupach urządzeń. Udostępniono interfejs webowy, dzięki któremu można wybrać urządzenia (lub ich grupy), na których chcemy przeprowadzić aktualizację i sprawdzić jakie aplikacje są zainstalowane. Ponieważ nie zawsze możliwe jest połączenie z wybranym urządzeniem, serwer przechowuje jego stan. W momencie uzyskania połączenia z urządzeniem serwer wykonuje zaplanowane aktualizacje.

\section{Zakres funkcfonalności}
\subsection{Wymagania funkcjonalne}
Platforma:
\begin{itemize}
  \item monitoruje, czy dane urządzenie jest dostępne
  \item monitoruje zainstalowane aplikacje oraz ich wersje na urządzeniach
  \item rejestruje urządzenia
  \item pozwala na definiowanie grup urządzeń
  \item umożliwia aktualizację i instalację aplikacji na pojedynczych urządzeniach
  \item umożliwia aktualizację i instalację aplikacji na grupach urządzeń
  \item umożliwia aktualizację i instalację za pomocą systemowych narzędzi takich jak apt
  \item webowy interfejs użytkownika
  \item umożliwia tworzenie paczek aplikacji dedykowanych dla platformy wraz ze skryptami instalacyjnymi
\end{itemize}
\subsection{Wymagania niefunkcjonalne}
\begin{itemize}
  \item Obsługa między 1000 - 10000 urządzeń
  \item Obsługa różnych architektur i systemów operacyjnych
  \item Prawidłowe działanie w przypadku zrywających się połączeń
  \item Wymagania stawiane dokumentacji:
    \begin{itemize}
      \item Podręcznik użytkownika
      \item Podręcznik instalacji i konfiguracji.
      \item Dokumentacja techniczna – dokumentacja kodu, opis testów.
    \end{itemize}
\end{itemize}

\section{Wybrane aspekty realizacji}
Właściwie wszystkie podsystemy zostały wykonane w Erlangu, aby zachować homogeniczne środowisko:
\begin{itemize}
  \item baza danych - \emph{mnesia}
  \item serwer http - \emph{mochiweb}
  \item backend
  \item aplikacja kliencka
\end{itemize}
Jedynym elementem, który używa innych technologii (\emph{ErlyDTL}, \emph{JavaScript}, \emph{jQuery}, \emph{CSS}) jest webowy interfejs użytkownika.

Komunikacja między urządzeniem klienckim i serwerem odbywa się przez własny protokół nad TCP. Jest on w łatwy sposób rozszerzalny, gdyby trzeba było dodać nowe funkcjonalności oraz zapewnia możliwość komunikacji z urządzeniami znajdującymi się za NATem.

\section{Organizacja pracy}

W trakcie prac podzieliliśmy się na dwa zespoły:
\begin{itemize}
\item zespół zajmujący się backendem w składzie Roman Janusz oraz Tomasz Kowal
\item zespół zajmujący się frontendem oraz bazą danych w składzie Przemysław Dąbek oraz Małgorzata Wielgus
\end{itemize}

Prace zostały podzielone na dwa etapy. W pierwszym dokonaliśmy przeglądu potrzebnych technologii oraz zbudowaliśmy prototypy. W drugim zaimplementowaliśmy właściwą funkcjonalność.
Pierwszy etap zakończył się w czerwcu 2011 roku. Pracę nad drugim zaczęliśmy w trakcie wakacji 2011 roku.

\section{Wyniki projektu}

Wynikiem projektu jest działające oprogramowanie wraz z dokumentacją. System został przetestowany na urządzeniu beagleboard. W dalszej kolejności należy przetestować działanie systemu na większej liczbie urządzeń. Można spróbować dodać obsługę innych menedżerów pakietów (\emph{yum}, \emph{port}). Należy również popracować nad bezpieczeństwem systemu. W tej chwili pozwala on podłączać się do systemu każdemu urządzeniu, które zostanie poprawnie skonfigurowane (brak autoryzacji). Pozwala również na aktualizację dowolnej aplikacji działającej na urządzeniu klienckim, a nie tylko tych zarządzanych za pomocą systemu.

\end{document}
