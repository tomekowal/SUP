% By zmienic jezyk na angielski/polski, dodaj opcje do klasy english lub polish
\documentclass[polish,12pt]{aghthesis}
\usepackage[utf8]{inputenc}
\usepackage{url}
\usepackage{listings}

\author{Przemysław Dąbek, Roman Janusz\\ Tomasz Kowal, Małgorzata Wielgus}

\title{Platforma do automatycznych aktualizacji oprogramowania na urządzeniach zdalnych - Przewodnik}

\supervisor{dr inż. Wojciech Turek}

\date{2012}

% Szablon przystosowany jest do druku dwustronnego,
\begin{document}

\maketitle

\section{Cel prac i wizja produktu}
\subsection{Opis problemu}
Problem: Duża ilość zdalnych urządzeń rozmieszczonych w terenie. Urządzenia komunikują się przez zawodną sieć. Na różnych urządzeniach działają różne wersje aplikacji. Zachodzi potrzeba aktualizacji oprogramowania na grupie urządzeń. Rozwiązanie: Platforma umożliwi monitorowanie oraz aktualizację wersji aplikacji na grupach urządzeń.

\subsection{Opis użytkownika i zewnętrznych podsystemów}
Użytkownikiem systemu jest osoba odpowiedzialna za oprogramowanie w systemie składającym się z urządzeń zdalnych. System składa się z dużej liczby tych urządzeń, które mogą mieć mocno ograniczone zasoby sprzętowe (np. \emph{Beagleboard}). Urządzenia mogą mieć różne architektury. Urządzenia będą łączyć się z serwerem głównym, do którego przesyłają zebrane dane. Komunikacja odbywa się za pomocą zawodnego połączenia (np. \emph{GSM}).

\subsection{Opis produktu}
Produkt składa się z serwera zawierającego repozytorium oprogramowania. Serwer ma za zadanie monitorować jakie wersje aplikacji znajdują się na konkretnych urządzeniach i grupach urządzeń. Udostępniono interfejs webowy, dzięki któremu można wybrać urządzenia (lub ich grupy), na których chcemy przeprowadzić aktualizację i sprawdzić jakie aplikacje są zainstalowane. Ponieważ nie zawsze możliwe jest połączenie z wybranym urządzeniem, serwer przechowuje jego stan. W momencie uzyskania połączenia z urządzeniem serwer wykonuje zaplanowane aktualizacje.

\section{Zakres funkcfonalności}
\subsection{Wymagania funkcjonalne}
Platforma:
\begin{itemize}
  \item monitoruje, czy dane urządzenie jest dostępne
  \item monitoruje zainstalowane aplikacje oraz ich wersje na urządzeniach
  \item rejestruje urządzenia
  \item pozwala na definiowanie grup urządzeń
  \item umożliwia aktualizację i instalację aplikacji na pojedynczych urządzeniach
  \item umożliwia aktualizację i instalację aplikacji na grupach urządzeń
  \item umożliwia aktualizację i instalację za pomocą systemowych narzędzi takich jak apt
  \item webowy interfejs użytkownika
  \item umożliwia tworzenie paczek aplikacji dedykowanych dla platformy wraz ze skryptami instalacyjnymi
\end{itemize}
\subsection{Wymagania niefunkcjonalne}
\begin{itemize}
  \item Obsługa między 1000 - 10000 urządzeń
  \item Obsługa różnych architektur i systemów operacyjnych
  \item Prawidłowe działanie w przypadku zrywających się połączeń
  \item Wymagania stawiane dokumentacji:
    \begin{itemize}
      \item Podręcznik użytkownika
      \item Podręcznik instalacji i konfiguracji.
      \item Dokumentacja techniczna – dokumentacja kodu, opis testów.
    \end{itemize}
\end{itemize}

\section{Wybrane aspekty realizacji}
Właściwie wszystkie podsystemy zostały wykonane w Erlangu, aby zachować homogeniczne środowisko:
\begin{itemize}
  \item baza danych - \emph{mnesia}
  \item serwer http - \emph{mochiweb}
  \item backend
  \item aplikacja kliencka
\end{itemize}
Jedynym elementem, który używa innych technologii (\emph{ErlyDTL}, \emph{JavaScript}, \emph{jQuery}, \emph{CSS}) jest webowy interfejs użytkownika.

Komunikacja między urządzeniem klienckim i serwerem odbywa się przez własny protokół nad TCP. Jest on w łatwy sposób rozszerzalny, gdyby trzeba było dodać nowe funkcjonalności oraz zapewnia możliwość komunikacji z urządzeniami znajdującymi się za NATem.

\section{Organizacja pracy}

W trakcie prac podzieliliśmy się na dwa zespoły:
\begin{itemize}
\item zespół zajmujący się backendem w składzie Roman Janusz oraz Tomasz Kowal
\item zespół zajmujący się frontendem oraz bazą danych w składzie Przemysław Dąbek oraz Małgorzata Wielgus
\end{itemize}

Prace zostały podzielone na dwa etapy. W pierwszym dokonaliśmy przeglądu potrzebnych technologii oraz zbudowaliśmy prototypy. W drugim zaimplementowaliśmy właściwą funkcjonalność.
Pierwszy etap zakończył się w czerwcu 2011 roku. Pracę nad drugim zaczęliśmy w trakcie wakacji 2011 roku.

\subsection*{Prace wykonane przez poszczególne osoby}

\begin{itemize}
  \item Roman Janusz
  \begin{itemize}
    \item Dokładne zapoznanie się z wzorcami OTP tworzenia oprogramowania w Erlangu - w szczególności \emph{application} i \emph{release}
    \item Zaprojektowanie ogólnej struktury oprogramowania przeznaczonego na urządzenia zdalne. Opracowanie modelu rozwoju oprogramowania z użyciem narzęcia \emph{rebar} i dodatkowych skryptów.
    \item Dokładne zapoznanie się ze strukturą pakietów \emph{deb}, działaniem menedżera pakietów \emph{dpkg}/\emph{apt} oraz metodami tworzenia pakietów i instalacji repozytorium. Rozpoznanie możliwości paczkowania oprogramowania tworzonego w erlangu z zachowaniem możliwości wykonywania hot-upgrade podczas aktualizacji pakietu.
    \item Zaprojektowanie sposobu dekompozycji erlangowego release'u w zestaw pakietów \emph{deb}. Stworzenie ogólnych skryptów (debian maintainer scripts) używanych podczas instalacji, aktualizacji i usuwania pakietów. Obsługa mechanizmu hot-upgrade podczas aktualizacji.
    \item Implementacja skryptów użytkownika do tworzenia pakietów \emph{deb} na podstawie erlangowego release'u wraz z ich konfiguracją.
    \item Integracja mechanizmu menedżera pakietów z platformą do automatycznych aktualizacji.
    \item Dokumentacja użytkownika i techniczna dotycząca sposobu wykorzystania menedżera pakietów \emph{deb} w projekcie.
  \end{itemize}

  \item Tomasz Kowal
  \begin{itemize}
    \item Testowanie możliwości serwera Mochiweb (ostatecznie użytego jako serwer http).
    \item Zbadanie różnych możliwości komunikacji klienta i backendu (\emph{gen\_tcp}, \emph{gen\_rcp}, użycie \emph{JSON}).
    \item Implementacja serwera przyjmującego zgłaszające się urządzenia klienckie.
    \item Szkielet niskopoziomowej komunikacji.
  \end{itemize}
  \item Przemysław Dąbek
    \begin{itemize}
      \item Przygotowanie interfejsu do operacji na bazie danych (mnesia).
      \item Zaprojektowanie webowego interfejsu użytkownika oraz logo.
      \item Implementacja logiki odpowiedzialnej za przetwarzanie żądań HTTP.
      \item Dodanie mechanizmu logowania (lager).
    \end{itemize}
  \item Małgorzata Wielgus
    \begin{itemize}
      \item Stworzenie prototypu interfejsu graficznego przy użyciu serwera \emph{Yaws}.
      \item Integracja backendu i frontendu serwera zarządzającego.
      \item Rozszerzenie interfejsu graficznego o funkcjonalność zlecania zadań urządzeniom.
      \item Testowanie różnych scenariuszy zakończenia \emph{joba}.
    \end{itemize}
\end{itemize}


\section{Wyniki projektu}

Wynikiem projektu jest działające oprogramowanie wraz z dokumentacją. Oprogramowanie składa się z dwóch podstawowych części: serwera zarządzającego oraz aplikacji klienckiej.

Serwer zarządzający po zainstalowaniu i uruchomieniu udostępnia repozytorium pakietów i obsługuje zgłaszające się urządzenia. Dzięki użyciu lekkich wątków erlangowych jest w stanie obsłużyć wiele urządzeń jednocześnie. Serwer udostępnia również webowy interfejs użytkownika dla administratora systemu umożliwiający operacje niezbędne z punktu widzenia zarządzania oprogramowaniem na dużej liczbie urządzeń. Pozwala on na przeglądanie stanów urządzeń i zainstalowanych na nich wersji aplikacji, grupowanie urządzeń oraz zlecanie aktualizacji urządzeniom.

Aplikację kliencką można zawrzeć w dowolnym \emph{releasie} erlangowym, co zapewnia możliwość prostego wykorzystania systemu do różnych zastosowań, w tym na urządzeniach mobilnych. System jest w pełni zintegrowany z menedżerem pakietów \emph{apt}. Można tworzyć paczki z programami erlangowymi w taki sposób, że poszczególne aplikacje znajdują się w osobnych pakietach, co jest dużą zaletą w stosunku do ręcznej aktualizacji, gdyż pobierane są jedynie paczki zawierające nowe wersje aplikacji.

Platforma została zaprojektowana z myślą o łatwości instalacji i użytkowania, a także rozwoju. Serwer zarządzający może zostać zainstalowany w dowolnym systemie operacyjnym z \emph{Erlangiem}. Instalacja odbywa się w jednym kroku, automatycznie zostają pobrane wszystkie zależności. Webowy interfejs użytkownika z natury może być obsługiwany z dowolnego urządzenia z przeglądarką internetową. Aktualnie system docelowy, na którym działa aplikacja kliencka musi być wyposażony w menedżer pakietów \emph{apt} oraz naturalnie emph{Erlanga}. Aplikację w systemie klienta można również w prosty sposób zainstalować w postaci paczki. System został przetestowany na urządzeniu \emph{beagleboard} z systemem operacyjnym \emph{debian}.

\subsection*{Możliwości dalszego rozwoju}
W dalszej kolejności można popracować nad integracją z pozostałymi menedżerami pakietów, takimi jak \emph{yum}, \emph{port}, \emph{opkg}. Protokół komunikacyjny jest na tyle elastyczny, że użytkownik w prosty sposób może definiować własne zadania, które z pomocą systemu zostaną przydzielone do wykonania na wielu urządzeniach. Kolejnym krokiem jest poprawa bezpieczeństwa przez zaimplementowanie autoryzacji zgłaszających się urządzeń oraz serwera, a także szyfrowania przesyłanych wiadomości. Aby umożliwić podgląd stanu systemu w dowolnej chwili warto dopisać aplikację na \emph{Androida} i \emph{iOSa}, które udostępniałyby te same funkcjonalności, co interfejs webowy. Dodatkowo wzbogaciłoby również dodanie dodatkowych informacji na temat urządzeń - na przykład położenia geograficznego.

\end{document}
